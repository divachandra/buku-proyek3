	\begin{figure}[H]
		\includegraphics[width=6cm]{figures/web/php.png}
		\centering
		\caption{Logo PHP}
	\end{figure}


\subsection{Apa itu PHP?}
PHP merupakan salah satu dari sekian banyak bahasa pemrograman web yang paling umum digunakan dalam pengembangan suatu web. Biasanya dalam implementasinya PHP sering digabungkan atau disisipkan dalam dokumen HTML. PHP memiliki kepanjangan yaitu PHP: Hypertext Preprocessor.Bahasa pemrograman ini bersifat server-side. Arti dari Server-side programming sendiri yaitu script/program tersebut akan dijalankan/diproses oleh server. 

\subsection{Sejarah PHP}
Pada awalnya PHP merupakan kependekan dari Personal Home Page (Situs personal). PHP pertama kali dibuat oleh Rasmus Lerdorf pada tahun 1995. Pada waktu itu PHP masih bernama Form Interpreted (FI), yang wujudnya berupa sekumpulan skrip yang digunakan untuk mengolah data formulir dari web.

Selanjutnya Rasmus merilis kode sumber tersebut untuk umum dan menamakannya PHP/FI. Dengan perilisan kode sumber ini menjadi sumber terbuka, maka banyak pemrogram yang tertarik untuk ikut mengembangkan PHP.

Pada November 1997, dirilis PHP/FI 2.0. Pada rilis ini, interpreter PHP sudah diimplementasikan dalam program C. Dalam rilis ini disertakan juga modul-modul ekstensi yang meningkatkan kemampuan PHP/FI secara signifikan.

Pada tahun 1997, sebuah perusahaan bernama Zend menulis ulang interpreter PHP menjadi lebih bersih, lebih baik, dan lebih cepat. Kemudian pada Juni 1998, perusahaan tersebut merilis interpreter baru untuk PHP dan meresmikan rilis tersebut sebagai PHP 3.0 dan singkatan PHP diubah menjadi akronim berulang PHP: Hypertext Preprocessing.

Pada pertengahan tahun 1999, Zend merilis interpreter PHP baru dan rilis tersebut dikenal dengan PHP 4.0. PHP 4.0 adalah versi PHP yang paling banyak dipakai pada awal abad ke-21. Versi ini banyak dipakai disebabkan kemampuannya untuk membangun aplikasi web kompleks tetapi tetap memiliki kecepatan dan stabilitas yang tinggi.

Pada Juni 2004, Zend merilis PHP 5.0. Dalam versi ini, inti dari interpreter PHP mengalami perubahan besar. Versi ini juga memasukkan model pemrograman berorientasi objek ke dalam PHP untuk menjawab perkembangan bahasa pemrograman ke arah paradigma berorientasi objek. Peladen web bawaan ditambahkan pada versi 5.4 untuk mempermudah pengembang menjalankan kode PHP tanpa menginstal peladen perangkat lunak.

Versi terbaru dan stabil dari bahasa pemograman PHP saat ini adalah versi 7.4.3 yang dirilis pada tanggal 20 Februari 2020

\subsection{Sekilas tentang pembuat PHP : Rasmus Lerdorf}
	\begin{figure}[H]
		\includegraphics[width=6cm]{figures/web/rasmuslerdorf.jpg}
		\centering
		\caption{Creator PHP Rasmus Lerdorf }
	\end{figure}
Rasmus Lerdorf merupakan seorang programmer yang berasal dari Denmark. Dia membuat dan membantu dalam hal pengkodean bahasa PHP, terutama pada 2 versi awal yang kemudian dikembangkan secara grup bersama dengan Jim Winstead (Yang membuat blo.gs), Stig Bakken, Shane Caraveo, Andi Gutmans, dan juga Zeev Suraski. sampai sekarang ia terus berkontribusi pada projek.

Lerdorf lahir di pulau disko di daerah Greenland dan kemudian pindah ke Denmark pada awal hidupnya. kelaurganya pindah dari Kanada ke Denmark pada tahun 1980, lalu pindah lagi ke kota King di Ontario pada tahun 1983. Dia lulus dari SMA King City pada tahun 1988, dan pada tahun 1993 lulus dari Universitas Waterloo dengan gelar  Bachelor of Applied Science di bidang teknik desain sistem. Dia juga ikut berkontribusi dalam Apache HTTP Server dan menambahkan clausa Limit pada mSQL DBMS. 

Dari Septermber 2002 sampai November 2009, Lerdorf bekerja di perusahaan Yahoo sebagai Infrastructure Architecture Engineer. Pada tahun 2010 kemudian bergabung ke perusahaan WePay untuk mengembangkan API (Application Programming Interface). dan pada tahun 2011 ia menjadi seorang konsultan untuk beberapa startup. Kemudian pada 22 Februari 2012 ia bergabung dengan Etsy, sebuah website e-commerce yang berfokus pada hal hal vintage. Dan pada tahun 2013, Rasmus bergabung dengan Jelastic sebagai Senior Advisor untuk membantu mereka mengembangkan teknologi baru.

Selain itu Lerdorf juga sering menjadi pembicara dalam konferensi open source di berbagai belahan dunia. Beberapa topik yang sering dia bahas diantaranya adalah security vulnerabilities dan juga tentang PHP.

Pada tahun 2003, ia di beri penghargaan oleh MIT Technology Review sebagai salah satu dari 100 inovator di dunia yang berada dibawah umur 35
\subsection{Website yang menggunakan PHP} 
	\begin{figure}[H]
		\includegraphics[width=8cm]{figures/web/popularphpsites.jpg}
		\centering
		\caption{Website dengan bahasa PHP}
	\end{figure}

\subsection{Contoh Kode PHP}
	\begin{figure}[H]
		\includegraphics[width=6cm]{figures/web/contohkodingphp.png}
		\centering
		\caption{Program Hello world yang ditulis dengan PHP}
	\end{figure}

\subsection{Kelebihan PHP}
\begin{itemize}
	\item Bahasa Pemrograman PHP dapat ditemukan di mana - mana.
	\item Proses pengembangan lebih mudah, karena komunitas yang bisa dibilang besar dan mendukung.
	\item PHP adalah bahasa scripting yang paling mudah karena memiliki referensi yang banyak dan lengkap.
	\item PHP adalah bahasa open source yang dapat digunakan di berbagai mesin (Linux, Unix, Macintosh, Windows).
	\item Ringkas dan ringan
	\item Maintenanace Mudah
\end{itemize}
\subsection{Kekurangan PHP}
\begin{itemize}
	\item Banyak kompetisi, karena PHP adalah bahasa pemrograman yang paling umum
	\item Terkesan kurang prestigious.
	\item Tidak ideal jika untuk pengembangan skala besar.
	\item PHP mempunyai kelemahan security tertentu.
\end{itemize}

\subsection{Referensi belajar PHP}
\begin{itemize}
	\item php.net
	\item sitepoint.com
	\item tutorialspoint.com/php/
\end{itemize}

	\begin{figure}[H]
		\includegraphics[width=12cm]{figures/web/phpframework.png}
		\centering
		\caption{Framework PHP}
	\end{figure}
\subsection{Apa itu Framework?}
PHP Framework adalah suatu kerangka keja yang telah terbentuk dan tersusun untuk memudahkan proses pengembangan website  secara profesional.

Namun perlu diketahui, bahwa PHP Framework memiliki perbedaannya tersendiri jika dibandingkan dengan sebuah CMS (Content Management System). Meskipun pada dasarnya mereka sama-sama memudahkan dalam hal yang sama yaitu pembuatan website. Tetapi untuk CMS  kita tidak perlu repot-repot dan pusing-pusing menulis script maupun sekumpulan kode. Dengan fitur CMS, semuanya telah dibuat instan dan kita hanya perlu sedikit mengatur bagian konten dan interface-nya saja.

Berbeda dengan Framework, kita tetap harus menuliskan script dan sekumpulan kode untuk dapat membangun sebuah web. 

\subsection{Mengapa harus menggunakan Framework?}
\begin{itemize}
	\item Mempercepat proses pengembangan web
	\item Kode yang terorganisir dan dapat digunakan terus menerus (reusable).
	\item Lebih mudah dalam proses maintenance.
	\item Lebih aman dalam hal sekuriti
	\item Menggunakan pola MVC (model - view - controller) yang memisahkan antara presentation dan logic
	\item Konsep web development modern seperti object oriented programming.

\end{itemize}

	\begin{figure}[H]
		\includegraphics[width=6cm]{figures/web/logocodeigniter.png}
		\centering
		\caption{Logo Codeigniter}
	\end{figure}

\subsection{Pengenalan Codeigniter}

CodeIgniter sendiri merupakan salah satu framework php yang bersifat aplikasi sumber terbuka(Open Source) dengan model MVC (Model, View, Controller) yang digunakan untuk mengembangkan situs web yang dinamis. CodeIgniter mempermudah proses pengembang web untuk membuat aplikasi web dengan waktu yang lebih cepat dan mudah dibandingkan dengan membuatnya dari awal. CodeIgniter dirilis pertama kali pada 28 Februari 2006. 

	\begin{figure}[H]
		\includegraphics[width=8cm]{figures/web/mvc.png}
		\centering
		\caption{MVC Concept}
	\end{figure}
\subsection{Konsep MVC (Model View dan Controller)}
Model View Controller merupakan suatu konsep yang cukup populer dalam mengembangkan suatu aplikasi web, Konsep MVC ini mencoba memisahkan pengembangan aplikasi berdasarkan komponen-komponen utama dalam membuat suatu aplikasi seperti misalkan bagian untuk pemrosesan/manipulasi data, tampilan , dan bagian untuk kontrol aplikasi. Terdapat 3 jenis komponen yang membangun suatu pola MVC dalam suatu aplikasi yaitu: 
\begin{enumerate}
	\item View, merupakan bagian yang akan ditampilkan dan dilihat oleh pengguna. Biasanya, yang ditampilkan adalah dokumen HTML yang diatur oleh controller. View berfungsi untuk menerima dan merepresentasikan data kepada pengguna. Bagian view tidak dapat memiliki akses pada bagian model.
	\item Model, Berhubungan langsung dengan proses CRUD (Create, Read, Update, Delete), serta search. Model juga menangani validasi dari bagian controller, tetapi berhubungan langsung dengan view.
	\item Controller, merupakan bagian yang menghubungkan model dan juga view, controller berperan dalam menerima data dari view kemudian memprosesnya untuk dikirim ke bagian model.
\end{enumerate}

\subsection{Kelebihan dan Kekurangan Codeigniter}
\begin{itemize}
	\item Kelebihan
\begin{itemize}
	\item Performa sangat cepat.
	\item Konfigurasi yang sangat minim
	\item Komunitas yang aktif dan mendukung
	\item Dokumentasi yang sangat lengkap
\end{itemize}
	\item Kekurangan
\begin{itemize}
	\item CodeIgniter tidak ditujukan untuk pembuatan web dengan skala besar.
	\item Tidak Adanya Editor Khusus.
\end{itemize}
\end{itemize}
\subsection{Instalasi Codeigniter}
\begin{itemize}

	\begin{figure}[H]
		\includegraphics[width=8cm]{figures/web/Xampp.png}
		\centering
		\caption{Tampilan Xampp}
	\end{figure}
	\item Download Xampp terlebih dahulu di https://www.apachefriends.org/
	\item Kemudian install Xampp
	\begin{figure}[H]
		\includegraphics[width=8cm]{figures/web/tampilanwebxampp.png}
		\centering
		\caption{Tampilan web Xampp jika berhasil instalasi}
	\end{figure}


	\begin{figure}[H]
		\includegraphics[width=8cm]{figures/web/websitecodeigniter.png}
		\centering
		\caption{Website Codeigniter}
	\end{figure}
	\item Kemudian, dilanjutkan dengan instalasi codeigniter dengan cara mendownload terlebih dahulu codeigniter-nya di https://codeigniter.com/en/download


	\begin{figure}[H]
		\includegraphics[width=8cm]{figures/web/ekstrakcodeigniter.png}
		\centering
		\caption{Ekstrak File}
	\end{figure}
	\item ekstrak file yang telah di download

	\item berikut adalah hasil ekstrak file
	\begin{figure}[H]
		\includegraphics[width=8cm]{figures/web/hasilekstrakcodeigniter.png}
		\centering
		\caption{Hasil Ekstrak File}
	\end{figure}

\end{itemize}

	\begin{figure}[H]
		\includegraphics[width=8cm]{figures/GambarAPI.png}
		\centering
		\caption{API}
	\end{figure}
\subsection{Pengenalan API}
API merupakan singkatan yang memiliki kepanjangan yaitu Application Programming Interface, API ini digunakan oleh developer untuk melakukan proses integrasi dua bagian dari satu atau lebeih aplikasi yang berbeda secara bersamaan. API terdiri dari hal-hal seperti function, protocols, dan juga tools. 

Tujuan dari penggunaan API sendiri adalah untuk mempercepat proses pengembangan dari suatu project dengan cara menyediakan beberapa function secara terpisah sehingga para pengembang tidak diperlukan untuk membuat function dengan fungsi tertentu dari awal, langsung pakai saja.

Penerapan API akan memiliki dampak yang signifikan jika fitur yang berkaitan sudah semakin kompleks yang pastinya akan membutuhkan waktu apabila harus memulainya dari awal hanya sekedar untuk membuat function yang memiliki fungsi yang sama dengan API yang telah ada

\subsection{Fungsi API}
API memiliki fungsi untuk menyediakan function yang lebih terstrukur serta terbaca dan mudah dipahami oleh seorang programmer. Hal ini merupakan hal yang krusial, terutama pada bagian editing serta pengembangan suatu projek

\subsection{Jenis-Jenis API}
\begin{itemize}
	\item Ownership Web API
	\item Communication Level API
	\item Web Service API
\end{itemize}