\subsubsection{Pembimbing Eksternal}

Selanjutnya kita akan membuat aplikasi Android untuk pembimbing eksternal (pembimbing perusahaan) menggunakan Android Studio. Jadi pastikan sebelumnya anda telah menginstall Android Studio. Pertama kita sesuaikan build.gradlenya menjadi seperti ini.
\lstinputlisting{src/actspotpembimbingeksternal/build.gradle}

Kemudian kita sesuaikan build.gradle kedua menjadi seperti ini.
\lstinputlisting{src/actspotpembimbingeksternal/build2.gradle}

Selanjutnya kita buat file model ConfirmAbsence.java di package model. Kemudian tambahkan script berikut di file tersebut.
\lstinputlisting[language=Java]{src/actspotpembimbingeksternal/model/ConfirmAbsence.java}

Setelah itu, kita buat file model ConfirmMahasiswa.java di package model. Kemudian tambahkan script berikut di file tersebut.
\lstinputlisting[language=Java]{src/actspotpembimbingeksternal/model/ConfirmMahasiswa.java}

Selanjutnya kita buat file model Login.java di package model. Kemudian tambahkan script berikut di file tersebut.
\lstinputlisting[language=Java]{src/actspotpembimbingeksternal/model/Login.java}

Setelah itu, kita buat file model Mahasiswa.java di package model. Kemudian tambahkan script berikut di file tersebut.
\lstinputlisting[language=Java]{src/actspotpembimbingeksternal/model/Mahasiswa.java}

Selanjutnya kita buat file model Notification.java di package model. Kemudian tambahkan script berikut di file tersebut.
\lstinputlisting[language=Java]{src/actspotpembimbingeksternal/model/Notification.java}

Setelah itu, kita buat file model ResponseConfirmAbsence.java di package model. Kemudian tambahkan script berikut di file tersebut.
\lstinputlisting[language=Java]{src/actspotpembimbingeksternal/model/ResponseConfirmAbsence.java}

Selanjutnya kita buat file model ResponseConfirmMahasiswa.java di package model. Kemudian tambahkan script berikut di file tersebut.
\lstinputlisting[language=Java]{src/actspotpembimbingeksternal/model/ResponseConfirmMahasiswa.java}

Setelah itu, kita buat file model ResponseInfo.java di package model. Kemudian tambahkan script berikut di file tersebut.
\lstinputlisting[language=Java]{src/actspotpembimbingeksternal/model/ResponseInfo.java}

Selanjutnya kita buat file model ResponseLogin.java di package model. Kemudian tambahkan script berikut di file tersebut.
\lstinputlisting[language=Java]{src/actspotpembimbingeksternal/model/ResponseLogin.java}

Setelah itu, kita buat file model ResponseNotification.java di package model. Kemudian tambahkan script berikut di file tersebut.
\lstinputlisting[language=Java]{src/actspotpembimbingeksternal/model/ResponseNotification.java}

Selanjutnya kita buat file model ResponseMahasiswa.java di package model. Kemudian tambahkan script berikut di file tersebut.
\lstinputlisting[language=Java]{src/actspotpembimbingeksternal/model/ResponseMahasiswa.java}

Setelah itu, kita buat file SharedPrefManager.java di package util. Kemudian tambahkan script berikut di file tersebut.
\lstinputlisting[language=Java]{src/actspotpembimbingeksternal/util/SharedPrefManager.java}

Selanjutnya kita buat file Urls.java di package util. Kemudian tambahkan script berikut di file tersebut.
\lstinputlisting[language=Java]{src/actspotpembimbingeksternal/util/Urls.java}

Setelah itu, kita buat file APIClient.java di package api. Kemudian tambahkan script berikut di file tersebut.
\lstinputlisting[language=Java]{src/actspotpembimbingeksternal/api/APIClient.java}

Selanjutnya kita buat file APIService.java di package api. Kemudian tambahkan script berikut di file tersebut.
\lstinputlisting[language=Java]{src/actspotpembimbingeksternal/api/APIService.java}

Setelah itu, kita buat file custom\_edittext.xml di package drawable. Kemudian tambahkan script berikut di file tersebut.
\lstinputlisting[language=xml]{src/actspotpembimbingeksternal/drawable/custom_edittext.xml}

Selanjutnya kita buat file state\_bottom\_navigation.xml di package drawable. Kemudian tambahkan script berikut di file tersebut.
\lstinputlisting[language=xml]{src/actspotpembimbingeksternal/drawable/state_bottom_navigation.xml}

Setelah itu, kita buat file menu\_bottom\_nav.xml di package menu. Kemudian tambahkan script berikut di file tersebut.
\lstinputlisting[language=xml]{src/actspotpembimbingeksternal/menu/menu_bottom_nav.xml}

Selanjutnya kita buat file activity\_detail\_mahasiswa.xml di package layout. Kemudian tambahkan script berikut di file tersebut.
\lstinputlisting[language=xml]{src/actspotpembimbingeksternal/layout/activity_detail_mahasiswa.xml}

Setelah itu, kita buat file DetailMahasiswaActivity.java di package activity. Kemudian tambahkan script berikut di file tersebut.
\lstinputlisting[language=Java]{src/actspotpembimbingeksternal/activity/DetailMahasiswaActivity.java}

Selanjutnya kita buat file activity\_login.xml di package layout. Kemudian tambahkan script berikut di file tersebut.
\lstinputlisting[language=xml]{src/actspotpembimbingeksternal/layout/activity_login.xml}

Setelah itu, kita buat file LoginActivity.java di package activity. Kemudian tambahkan script berikut di file tersebut.
\lstinputlisting[language=Java]{src/actspotpembimbingeksternal/activity/LoginActivity.java}

Selanjutnya kita buat file activity\_notice.xml di package layout. Kemudian tambahkan script berikut di file tersebut.
\lstinputlisting[language=xml]{src/actspotpembimbingeksternal/layout/activity_notice.xml}

Setelah itu, kita buat file NoticeActivity.java di package activity. Kemudian tambahkan script berikut di file tersebut.
\lstinputlisting[language=Java]{src/actspotpembimbingeksternal/activity/NoticeActivity.java}

Selanjutnya kita buat file activity\_main.xml di package layout. Kemudian tambahkan script berikut di file tersebut.
\lstinputlisting[language=xml]{src/actspotpembimbingeksternal/layout/activity_main.xml}

Setelah itu, kita buat file MainActivity.java di package activity. Kemudian tambahkan script berikut di file tersebut.
\lstinputlisting[language=Java]{src/actspotpembimbingeksternal/activity/MainActivity.java}

Selanjutnya kita buat file activity\_password.xml di package layout. Kemudian tambahkan script berikut di file tersebut.
\lstinputlisting[language=xml]{src/actspotpembimbingeksternal/layout/activity_password.xml}

Setelah itu, kita buat file PasswordActivity.java di package activity. Kemudian tambahkan script berikut di file tersebut.
\lstinputlisting[language=Java]{src/actspotpembimbingeksternal/activity/PasswordActivity.java}

Selanjutnya kita buat file activity\_profile.xml di package layout. Kemudian tambahkan script berikut di file tersebut.
\lstinputlisting[language=xml]{src/actspotpembimbingeksternal/layout/activity_profile.xml}

Setelah itu, kita buat file ProfileActivity.java di package activity. Kemudian tambahkan script berikut di file tersebut.
\lstinputlisting[language=Java]{src/actspotpembimbingeksternal/activity/ProfileActivity.java}

Selanjutnya kita buat file activity\_register.xml di package layout. Kemudian tambahkan script berikut di file tersebut.
\lstinputlisting[language=xml]{src/actspotpembimbingeksternal/layout/activity_register.xml}

Setelah itu, kita buat file RegisterActivity.java di package activity. Kemudian tambahkan script berikut di file tersebut.
\lstinputlisting[language=Java]{src/actspotpembimbingeksternal/activity/RegisterActivity.java}

Selanjutnya kita buat file activity\_text.xml di package layout. Kemudian tambahkan script berikut di file tersebut.
\lstinputlisting[language=xml]{src/actspotpembimbingeksternal/layout/activity_text.xml}

Setelah itu, kita buat file TextActivity.java di package activity. Kemudian tambahkan script berikut di file tersebut.
\lstinputlisting[language=Java]{src/actspotpembimbingeksternal/activity/TextActivity.java}

Selanjutnya kita buat file item\_confirm\_notification.xml di package layout. Kemudian tambahkan script berikut di file tersebut.
\lstinputlisting[language=xml]{src/actspotpembimbingeksternal/layout/item_confirm_notification.xml}

Setelah itu, kita buat file ConfirmAbsenceAdapter.java di package adapter. Kemudian tambahkan script berikut di file tersebut.
\lstinputlisting[language=Java]{src/actspotpembimbingeksternal/adapter/ConfirmAbsenceAdapter.java}

Selanjutnya kita buat file item\_confirm\_notification.xml di package layout. Kemudian tambahkan script berikut di file tersebut.
\lstinputlisting[language=xml]{src/actspotpembimbingeksternal/layout/item_confirm_notification.xml}

Setelah itu, kita buat file ConfirmMahasiswaAdapter.java di package adapter. Kemudian tambahkan script berikut di file tersebut.
\lstinputlisting[language=Java]{src/actspotpembimbingeksternal/adapter/ConfirmMahasiswaAdapter.java}

Selanjutnya kita buat file item\_mahasiswa.xml di package layout. Kemudian tambahkan script berikut di file tersebut.
\lstinputlisting[language=xml]{src/actspotpembimbingeksternal/layout/item_mahasiswa.xml}

Setelah itu, kita buat file MahasiswaAdapter.java di package adapter. Kemudian tambahkan script berikut di file tersebut.
\lstinputlisting[language=Java]{src/actspotpembimbingeksternal/adapter/MahasiswaAdapter.java}

Selanjutnya kita buat file item\_notification.xml di package layout. Kemudian tambahkan script berikut di file tersebut.
\lstinputlisting[language=xml]{src/actspotpembimbingeksternal/layout/item_notification.xml}

Setelah itu, kita buat file NotificationAdapter.java di package adapter. Kemudian tambahkan script berikut di file tersebut.
\lstinputlisting[language=Java]{src/actspotpembimbingeksternal/adapter/NotificationAdapter.java}

Setelah itu, kita buat file PageAdapter.java di package adapter. Kemudian tambahkan script berikut di file tersebut.
\lstinputlisting[language=Java]{src/actspotpembimbingeksternal/adapter/PageAdapter.java}

Selanjutnya kita buat file fragment\_absence\_confirm.xml di package layout. Kemudian tambahkan script berikut di file tersebut.
\lstinputlisting[language=xml]{src/actspotpembimbingeksternal/layout/fragment_absence_confirm.xml}

Setelah itu, kita buat file AbsenceConfirmFragment.java di package fragment. Kemudian tambahkan script berikut di file tersebut.
\lstinputlisting[language=Java]{src/actspotpembimbingeksternal/fragment/AbsenceConfirmFragment.java}

Selanjutnya kita buat file fragment\_confirmation.xml di package layout. Kemudian tambahkan script berikut di file tersebut.
\lstinputlisting[language=xml]{src/actspotpembimbingeksternal/layout/fragment_confirmation.xml}

Setelah itu, kita buat file ConfirmationFragment.java di package fragment. Kemudian tambahkan script berikut di file tersebut.
\lstinputlisting[language=Java]{src/actspotpembimbingeksternal/fragment/ConfirmationFragment.java}

Selanjutnya kita buat file fragment\_mahasiswa\_confirm.xml di package layout. Kemudian tambahkan script berikut di file tersebut.
\lstinputlisting[language=xml]{src/actspotpembimbingeksternal/layout/fragment_mahasiswa_confirm.xml}

Setelah itu, kita buat file MahasiswaConfirmFragment.java di package fragment. Kemudian tambahkan script berikut di file tersebut.
\lstinputlisting[language=Java]{src/actspotpembimbingeksternal/fragment/MahasiswaConfirmFragment.java}

Selanjutnya kita buat file fragment\_notification.xml di package layout. Kemudian tambahkan script berikut di file tersebut.
\lstinputlisting[language=xml]{src/actspotpembimbingeksternal/layout/fragment_notification.xml}

Setelah itu, kita buat file NotificationFragment.java di package fragment. Kemudian tambahkan script berikut di file tersebut.
\lstinputlisting[language=Java]{src/actspotpembimbingeksternal/fragment/NotificationFragment.java}

Selanjutnya kita buat file fragment\_profile.xml di package layout. Kemudian tambahkan script berikut di file tersebut.
\lstinputlisting[language=xml]{src/actspotpembimbingeksternal/layout/fragment_profile.xml}

Setelah itu, kita buat file ProfileFragment.java di package fragment. Kemudian tambahkan script berikut di file tersebut.
\lstinputlisting[language=Java]{src/actspotpembimbingeksternal/fragment/ProfileFragment.java}

Selanjutnya kita buat file fragment\_mahasiswa.xml di package layout. Kemudian tambahkan script berikut di file tersebut.
\lstinputlisting[language=xml]{src/actspotpembimbingeksternal/layout/fragment_mahasiswa.xml}

Setelah itu, kita buat file MahasiswaFragment.java di package fragment. Kemudian tambahkan script berikut di file tersebut.
\lstinputlisting[language=Java]{src/actspotpembimbingeksternal/fragment/MahasiswaFragment.java}

Selanjutnya tambahkan script berikut di file tersebut pada file AndroidManifest.xml
\lstinputlisting[language=xml]{src/actspotpembimbingeksternal/AndroidManifest.xml}