\subsubsection{Pembimbing Internal}

Selanjutnya kita akan membuat aplikasi Android untuk pembimbing internal (dosen) menggunakan Android Studio. Jadi pastikan sebelumnya anda telah menginstall Android Studio. Pertama kita sesuaikan build.gradlenya menjadi seperti ini.
\lstinputlisting{src/actspotpembimbinginternal/build.gradle}

Kemudian kita sesuaikan build.gradle kedua menjadi seperti ini.
\lstinputlisting{src/actspotpembimbinginternal/build2.gradle}

Selanjutnya kita buat file model Login.java di package model. Kemudian tambahkan script berikut di file tersebut.
\lstinputlisting[language=Java]{src/actspotpembimbinginternal/model/Login.java}

Setelah itu, kita buat file model Mahasiswa.java di package model. Kemudian tambahkan script berikut di file tersebut.
\lstinputlisting[language=Java]{src/actspotpembimbinginternal/model/Mahasiswa.java}

Selanjutnya kita buat file model Notification.java di package model. Kemudian tambahkan script berikut di file tersebut.
\lstinputlisting[language=Java]{src/actspotpembimbinginternal/model/Notification.java}

Setelah itu, kita buat file model Progress.java di package model. Kemudian tambahkan script berikut di file tersebut.
\lstinputlisting[language=Java]{src/actspotpembimbinginternal/model/Progress.java}

Selanjutnya kita buat file model ProgressPerDay.java di package model. Kemudian tambahkan script berikut di file tersebut.
\lstinputlisting[language=Java]{src/actspotpembimbinginternal/model/ProgressPerDay.java}

Setelah itu, kita buat file model ResponseInfo.java di package model. Kemudian tambahkan script berikut di file tersebut.
\lstinputlisting[language=Java]{src/actspotpembimbinginternal/model/ResponseInfo.java}

Selanjutnya kita buat file model ResponseLogin.java di package model. Kemudian tambahkan script berikut di file tersebut.
\lstinputlisting[language=Java]{src/actspotpembimbinginternal/model/ResponseLogin.java}

Setelah itu, kita buat file model ResponseNotification.java di package model. Kemudian tambahkan script berikut di file tersebut.
\lstinputlisting[language=Java]{src/actspotpembimbinginternal/model/ResponseNotification.java}

Selanjutnya kita buat file model ResponseProgress.java di package model. Kemudian tambahkan script berikut di file tersebut.
\lstinputlisting[language=Java]{src/actspotpembimbinginternal/model/ResponseProgress.java}

Setelah itu, kita buat file model ResponseProgressPerDay.java di package model. Kemudian tambahkan script berikut di file tersebut.
\lstinputlisting[language=Java]{src/actspotpembimbinginternal/model/ResponseProgressPerDay.java}

Selanjutnya kita buat file SharedPrefManager.java di package util. Kemudian tambahkan script berikut di file tersebut.
\lstinputlisting[language=Java]{src/actspotpembimbinginternal/util/SharedPrefManager.java}

Setelah itu, kita buat file Urls.java di package util. Kemudian tambahkan script berikut di file tersebut.
\lstinputlisting[language=Java]{src/actspotpembimbinginternal/util/Urls.java}

Selanjutnya kita buat file APIClient.java di package api. Kemudian tambahkan script berikut di file tersebut.
\lstinputlisting[language=Java]{src/actspotpembimbinginternal/api/APIClient.java}

Setelah itu, kita buat file APIService.java di package api. Kemudian tambahkan script berikut di file tersebut.
\lstinputlisting[language=Java]{src/actspotpembimbinginternal/api/APIService.java}

Selanjutnya kita buat file custom\_edittext.xml di package drawable. Kemudian tambahkan script berikut di file tersebut.
\lstinputlisting[language=xml]{src/actspotpembimbinginternal/drawable/custom_edittext.xml}

Setelah itu, kita buat file state\_bottom\_navigation.xml di package drawable. Kemudian tambahkan script berikut di file tersebut.
\lstinputlisting[language=xml]{src/actspotpembimbinginternal/drawable/state_bottom_navigation.xml}

Selanjutnya kita buat file menu\_bottom\_nav.xml di package menu. Kemudian tambahkan script berikut di file tersebut.
\lstinputlisting[language=xml]{src/actspotpembimbinginternal/menu/menu_bottom_nav.xml}

Setelah itu, kita buat file menu\_progress.xml di package menu. Kemudian tambahkan script berikut di file tersebut.
\lstinputlisting[language=xml]{src/actspotpembimbinginternal/menu/menu_progress.xml}

Selanjutnya kita buat file activity\_detail\_progress.xml di package layout. Kemudian tambahkan script berikut di file tersebut.
\lstinputlisting[language=xml]{src/actspotpembimbinginternal/layout/activity_detail_progress.xml}

Setelah itu, kita buat file DetailProgressActivity.java di package activity. Kemudian tambahkan script berikut di file tersebut.
\lstinputlisting[language=Java]{src/actspotpembimbinginternal/activity/DetailProgressActivity.java}

Selanjutnya kita buat file activity\_login.xml di package layout. Kemudian tambahkan script berikut di file tersebut.
\lstinputlisting[language=xml]{src/actspotpembimbinginternal/layout/activity_login.xml}

Setelah itu, kita buat file LoginActivity.java di package activity. Kemudian tambahkan script berikut di file tersebut.
\lstinputlisting[language=Java]{src/actspotpembimbinginternal/activity/LoginActivity.java}

Selanjutnya kita buat file activity\_main.xml di package layout. Kemudian tambahkan script berikut di file tersebut.
\lstinputlisting[language=xml]{src/actspotpembimbinginternal/layout/activity_main.xml}

Setelah itu, kita buat file MainActivity.java di package activity. Kemudian tambahkan script berikut di file tersebut.
\lstinputlisting[language=Java]{src/actspotpembimbinginternal/activity/MainActivity.java}

Selanjutnya kita buat file activity\_password.xml di package layout. Kemudian tambahkan script berikut di file tersebut.
\lstinputlisting[language=xml]{src/actspotpembimbinginternal/layout/activity_password.xml}

Setelah itu, kita buat file PasswordActivity.java di package activity. Kemudian tambahkan script berikut di file tersebut.
\lstinputlisting[language=Java]{src/actspotpembimbinginternal/activity/PasswordActivity.java}

Selanjutnya kita buat file activity\_profile.xml di package layout. Kemudian tambahkan script berikut di file tersebut.
\lstinputlisting[language=xml]{src/actspotpembimbinginternal/layout/activity_profile.xml}

Setelah itu, kita buat file ProfileActivity.java di package activity. Kemudian tambahkan script berikut di file tersebut.
\lstinputlisting[language=Java]{src/actspotpembimbinginternal/activity/ProfileActivity.java}

Selanjutnya kita buat file activity\_progress.xml di package layout. Kemudian tambahkan script berikut di file tersebut.
\lstinputlisting[language=xml]{src/actspotpembimbinginternal/layout/activity_progress.xml}

Setelah itu, kita buat file ProgressActivity.java di package activity. Kemudian tambahkan script berikut di file tersebut.
\lstinputlisting[language=Java]{src/actspotpembimbinginternal/activity/ProgressActivity.java}

Selanjutnya kita buat file activity\_text.xml di package layout. Kemudian tambahkan script berikut di file tersebut.
\lstinputlisting[language=xml]{src/actspotpembimbinginternal/layout/activity_text.xml}

Setelah itu, kita buat file TextActivity.java di package activity. Kemudian tambahkan script berikut di file tersebut.
\lstinputlisting[language=Java]{src/actspotpembimbinginternal/activity/TextActivity.java}

Selanjutnya kita buat file item\_detail\_progress.xml di package layout. Kemudian tambahkan script berikut di file tersebut.
\lstinputlisting[language=xml]{src/actspotpembimbinginternal/layout/item_detail_progress.xml}

Setelah itu, kita buat file DetailProgressAdapter.java di package adapter. Kemudian tambahkan script berikut di file tersebut.
\lstinputlisting[language=Java]{src/actspotpembimbinginternal/adapter/DetailProgressAdapter.java}

Selanjutnya kita buat file item\_mahasiswa.xml di package layout. Kemudian tambahkan script berikut di file tersebut.
\lstinputlisting[language=xml]{src/actspotpembimbinginternal/layout/item_mahasiswa.xml}

Setelah itu, kita buat file MahasiswaAdapter.java di package adapter. Kemudian tambahkan script berikut di file tersebut.
\lstinputlisting[language=Java]{src/actspotpembimbinginternal/adapter/MahasiswaAdapter.java}

Selanjutnya kita buat file item\_notification.xml di package layout. Kemudian tambahkan script berikut di file tersebut.
\lstinputlisting[language=xml]{src/actspotpembimbinginternal/layout/item_notification.xml}

Setelah itu, kita buat file NotificationAdapter.java di package adapter. Kemudian tambahkan script berikut di file tersebut.
\lstinputlisting[language=Java]{src/actspotpembimbinginternal/adapter/NotificationAdapter.java}

Selanjutnya kita buat file item\_progress.xml di package layout. Kemudian tambahkan script berikut di file tersebut.
\lstinputlisting[language=xml]{src/actspotpembimbinginternal/layout/item_progress.xml}

Setelah itu, kita buat file ProgressAdapter.java di package adapter. Kemudian tambahkan script berikut di file tersebut.
\lstinputlisting[language=Java]{src/actspotpembimbinginternal/adapter/ProgressAdapter.java}

Selanjutnya kita buat file fragment\_notification.xml di package layout. Kemudian tambahkan script berikut di file tersebut.
\lstinputlisting[language=xml]{src/actspotpembimbinginternal/layout/fragment_notification.xml}

Setelah itu, kita buat file NotificationFragment.java di package fragment. Kemudian tambahkan script berikut di file tersebut.
\lstinputlisting[language=Java]{src/actspotpembimbinginternal/fragment/NotificationFragment.java}

Selanjutnya kita buat file fragment\_profile.xml di package layout. Kemudian tambahkan script berikut di file tersebut.
\lstinputlisting[language=xml]{src/actspotpembimbinginternal/layout/fragment_profile.xml}

Setelah itu, kita buat file ProfileFragment.java di package fragment. Kemudian tambahkan script berikut di file tersebut.
\lstinputlisting[language=Java]{src/actspotpembimbinginternal/fragment/ProfileFragment.java}

Selanjutnya kita buat file fragment\_mahasiswa.xml di package layout. Kemudian tambahkan script berikut di file tersebut.
\lstinputlisting[language=xml]{src/actspotpembimbinginternal/layout/fragment_mahasiswa.xml}

Setelah itu, kita buat file MahasiswaFragment.java di package fragment. Kemudian tambahkan script berikut di file tersebut.
\lstinputlisting[language=Java]{src/actspotpembimbinginternal/fragment/MahasiswaFragment.java}

Selanjutnya tambahkan script berikut di file tersebut pada file AndroidManifest.xml
\lstinputlisting[language=xml]{src/actspotpembimbinginternal/AndroidManifest.xml}