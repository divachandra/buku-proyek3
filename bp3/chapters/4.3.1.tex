\subsubsection{Mahasiswa}

Selanjutnya kita akan membuat aplikasi Android untuk mahasiswa menggunakan Android Studio. Jadi pastikan sebelumnya anda telah menginstall Android Studio. Pertama kita sesuaikan build.gradlenya menjadi seperti ini.
\lstinputlisting{src/actspotmahasiswa/build.gradle}

Kemudian kita sesuaikan build.gradle kedua menjadi seperti ini.
\lstinputlisting{src/actspotmahasiswa/build2.gradle}

Selanjutnya kita buat file model Kegiatan.java di package model. Kemudian tambahkan script berikut di file tersebut.
\lstinputlisting[language=Java]{src/actspotmahasiswa/model/Kegiatan.java}

Setelah itu, kita buat file model Login.java di package model. Kemudian tambahkan script berikut di file tersebut.
\lstinputlisting[language=Java]{src/actspotmahasiswa/model/Login.java}

Selanjutnya kita buat file model Notification.java di package model. Kemudian tambahkan script berikut di file tersebut.
\lstinputlisting[language=Java]{src/actspotmahasiswa/model/Notification.java}

Setelah itu, kita buat file model Pembimbing.java di package model. Kemudian tambahkan script berikut di file tersebut.
\lstinputlisting[language=Java]{src/actspotmahasiswa/model/Pembimbing.java}

Selanjutnya kita buat file model Kegiatan.java di package model. Kemudian tambahkan script berikut di file tersebut.
\lstinputlisting[language=Java]{src/actspotmahasiswa/model/Progress.java}

Setelah itu, kita buat file model ProgressPerDay.java di package model. Kemudian tambahkan script berikut di file tersebut.
\lstinputlisting[language=Java]{src/actspotmahasiswa/model/ProgressPerDay.java}

Selanjutnya kita buat file model ResponseInfo.java di package model. Kemudian tambahkan script berikut di file tersebut.
\lstinputlisting[language=Java]{src/actspotmahasiswa/model/ResponseInfo.java}

Setelah itu, kita buat file model ResponseKegiatan.java di package model. Kemudian tambahkan script berikut di file tersebut.
\lstinputlisting[language=Java]{src/actspotmahasiswa/model/ResponseKegiatan.java}

Setelah itu, kita buat file model ResponseLogin.java di package model. Kemudian tambahkan script berikut di file tersebut.
\lstinputlisting[language=Java]{src/actspotmahasiswa/model/ResponseLogin.java}

Selanjutnya kita buat file model ResponseInfo.java di package model. Kemudian tambahkan script berikut di file tersebut.
\lstinputlisting[language=Java]{src/actspotmahasiswa/model/ResponseNotification.java}

Setelah itu, kita buat file model ResponsePembimbing.java di package model. Kemudian tambahkan script berikut di file tersebut.
\lstinputlisting[language=Java]{src/actspotmahasiswa/model/ResponsePembimbing.java}

Selanjutnya kita buat file model ResponseProgress.java di package model. Kemudian tambahkan script berikut di file tersebut.
\lstinputlisting[language=Java]{src/actspotmahasiswa/model/ResponseProgress.java}

Setelah itu, kita buat file model ResponseProgressPerDay.java di package model. Kemudian tambahkan script berikut di file tersebut.
\lstinputlisting[language=Java]{src/actspotmahasiswa/model/ResponseProgressPerDay.java}

Selanjutnya kita buat file MahasiswaFileProvider.java di package util. Kemudian tambahkan script berikut di file tersebut.
\lstinputlisting[language=Java]{src/actspotmahasiswa/util/MahasiswaFileProvider.java}

Setelah itu, kita buat file SharedPrefManager.java di package util. Kemudian tambahkan script berikut di file tersebut.
\lstinputlisting[language=Java]{src/actspotmahasiswa/util/SharedPrefManager.java}

Selanjutnya kita buat file Urls.java di package util. Kemudian tambahkan script berikut di file tersebut.
\lstinputlisting[language=Java]{src/actspotmahasiswa/util/Urls.java}

Setelah itu, kita buat file APIClient.java di package api. Kemudian tambahkan script berikut di file tersebut.
\lstinputlisting[language=Java]{src/actspotmahasiswa/api/APIClient.java}

Selanjutnya kita buat file APIService.java di package api. Kemudian tambahkan script berikut di file tersebut.
\lstinputlisting[language=Java]{src/actspotmahasiswa/api/APIService.java}

Setelah itu, kita buat file custom\_edittext.xml di package drawable. Kemudian tambahkan script berikut di file tersebut.
\lstinputlisting[language=xml]{src/actspotmahasiswa/drawable/custom_edittext.xml}

Selanjutnya kita buat file state\_bottom\_navigation.xml di package drawable. Kemudian tambahkan script berikut di file tersebut.
\lstinputlisting[language=xml]{src/actspotmahasiswa/drawable/state_bottom_navigation.xml}

Setelah itu, kita buat file menu\_bottom\_nav.xml di package menu. Kemudian tambahkan script berikut di file tersebut.
\lstinputlisting[language=xml]{src/actspotmahasiswa/menu/menu_bottom_nav.xml}

Selanjutnya, kita buat file menu\_progress.xml di package menu. Kemudian tambahkan script berikut di file tersebut.
\lstinputlisting[language=xml]{src/actspotmahasiswa/menu/menu_progress.xml}

Setelah itu, kita buat file activity\_company.xml di package layout. Kemudian tambahkan script berikut di file tersebut.
\lstinputlisting[language=xml]{src/actspotmahasiswa/layout/activity_company.xml}

Selanjutnya kita buat file CompanyActivity.java di package activity. Kemudian tambahkan script berikut di file tersebut.
\lstinputlisting[language=Java]{src/actspotmahasiswa/activity/CompanyActivity.java}

Setelah itu, kita buat file activity\_detail\_progress.xml di package layout. Kemudian tambahkan script berikut di file tersebut.
\lstinputlisting[language=xml]{src/actspotmahasiswa/layout/activity_detail_progress.xml}

Selanjutnya kita buat file DetailProgressActivity.java di package activity. Kemudian tambahkan script berikut di file tersebut.
\lstinputlisting[language=Java]{src/actspotmahasiswa/activity/DetailProgressActivity.java}
%%%%%
Setelah itu, kita buat file activity\_login.xml di package layout. Kemudian tambahkan script berikut di file tersebut.
\lstinputlisting[language=xml]{src/actspotmahasiswa/layout/activity_login.xml}

Selanjutnya kita buat file LoginActivity.java di package activity. Kemudian tambahkan script berikut di file tersebut.
\lstinputlisting[language=Java]{src/actspotmahasiswa/activity/LoginActivity.java}

Setelah itu, kita buat file activity\_main.xml di package layout. Kemudian tambahkan script berikut di file tersebut.
\lstinputlisting[language=xml]{src/actspotmahasiswa/layout/activity_main.xml}

Selanjutnya kita buat file MainActivity.java di package activity. Kemudian tambahkan script berikut di file tersebut.
\lstinputlisting[language=Java]{src/actspotmahasiswa/activity/MainActivity.java}

Setelah itu, kita buat file activity\_notice.xml di package layout. Kemudian tambahkan script berikut di file tersebut.
\lstinputlisting[language=xml]{src/actspotmahasiswa/layout/activity_notice.xml}

Selanjutnya kita buat file NoticeActivity.java di package activity. Kemudian tambahkan script berikut di file tersebut.
\lstinputlisting[language=Java]{src/actspotmahasiswa/activity/NoticeActivity.java}

Setelah itu, kita buat file activity\_password.xml di package layout. Kemudian tambahkan script berikut di file tersebut.
\lstinputlisting[language=xml]{src/actspotmahasiswa/layout/activity_password.xml}

Selanjutnya kita buat file PasswordActivity.java di package activity. Kemudian tambahkan script berikut di file tersebut.
\lstinputlisting[language=Java]{src/actspotmahasiswa/activity/PasswordActivity.java}

Setelah itu, kita buat file activity\_profile.xml di package layout. Kemudian tambahkan script berikut di file tersebut.
\lstinputlisting[language=xml]{src/actspotmahasiswa/layout/activity_profile.xml}

Selanjutnya kita buat file ProfileActivity.java di package activity. Kemudian tambahkan script berikut di file tersebut.
\lstinputlisting[language=Java]{src/actspotmahasiswa/activity/ProfileActivity.java}

Setelah itu, kita buat file activity\_register.xml di package layout. Kemudian tambahkan script berikut di file tersebut.
\lstinputlisting[language=xml]{src/actspotmahasiswa/layout/activity_register.xml}

Selanjutnya kita buat file RegisterActivity.java di package activity. Kemudian tambahkan script berikut di file tersebut.
\lstinputlisting[language=Java]{src/actspotmahasiswa/activity/RegisterActivity.java}

Setelah itu, kita buat file activity\_spinner.xml di package layout. Kemudian tambahkan script berikut di file tersebut.
\lstinputlisting[language=xml]{src/actspotmahasiswa/layout/activity_spinner.xml}

Selanjutnya kita buat file SpinnerActivity.java di package activity. Kemudian tambahkan script berikut di file tersebut.
\lstinputlisting[language=Java]{src/actspotmahasiswa/activity/SpinnerActivity.java}

Setelah itu, kita buat file activity\_text.xml di package layout. Kemudian tambahkan script berikut di file tersebut.
\lstinputlisting[language=xml]{src/actspotmahasiswa/layout/activity_text.xml}

Selanjutnya kita buat file TextActivity.java di package activity. Kemudian tambahkan script berikut di file tersebut.
\lstinputlisting[language=Java]{src/actspotmahasiswa/activity/TextActivity.java}

Setelah itu, kita buat file item\_detail\_progress.xml di package layout. Kemudian tambahkan script berikut di file tersebut.
\lstinputlisting[language=xml]{src/actspotmahasiswa/layout/item_detail_progress.xml}

Selanjutnya kita buat file DetailProgressAdapter.java di package adapter. Kemudian tambahkan script berikut di file tersebut.
\lstinputlisting[language=Java]{src/actspotmahasiswa/adapter/DetailProgressAdapter.java}

Setelah itu, kita buat file item\_notification.xml di package layout. Kemudian tambahkan script berikut di file tersebut.
\lstinputlisting[language=xml]{src/actspotmahasiswa/layout/item_notification.xml}

Selanjutnya kita buat file NotificationAdapter.java di package adapter. Kemudian tambahkan script berikut di file tersebut.
\lstinputlisting[language=Java]{src/actspotmahasiswa/adapter/NotificationAdapter.java}

Setelah itu, kita buat file item\_progress.xml di package layout. Kemudian tambahkan script berikut di file tersebut.
\lstinputlisting[language=xml]{src/actspotmahasiswa/layout/item_progress.xml}

Selanjutnya kita buat file ProgressAdapter.java di package adapter. Kemudian tambahkan script berikut di file tersebut.
\lstinputlisting[language=Java]{src/actspotmahasiswa/adapter/ProgressAdapter.java}

Setelah itu, kita buat file fragment\_absence.xml di package layout. Kemudian tambahkan script berikut di file tersebut.
\lstinputlisting[language=xml]{src/actspotmahasiswa/layout/fragment_absence.xml}

Selanjutnya kita buat file AbsenceFragment.java di package fragment. Kemudian tambahkan script berikut di file tersebut.
\lstinputlisting[language=Java]{src/actspotmahasiswa/fragment/AbsenceFragment.java}

Setelah itu, kita buat file fragment\_notification.xml di package layout. Kemudian tambahkan script berikut di file tersebut.
\lstinputlisting[language=xml]{src/actspotmahasiswa/layout/fragment_notification.xml}

Selanjutnya kita buat file NotificationFragment.java di package fragment. Kemudian tambahkan script berikut di file tersebut.
\lstinputlisting[language=Java]{src/actspotmahasiswa/fragment/NotificationFragment.java}

Setelah itu, kita buat file fragment\_profile.xml di package layout. Kemudian tambahkan script berikut di file tersebut.
\lstinputlisting[language=xml]{src/actspotmahasiswa/layout/fragment_profile.xml}

Selanjutnya kita buat file ProfileFragment.java di package fragment. Kemudian tambahkan script berikut di file tersebut.
\lstinputlisting[language=Java]{src/actspotmahasiswa/fragment/ProfileFragment.java}

Setelah itu, kita buat file fragment\_progress.xml di package layout. Kemudian tambahkan script berikut di file tersebut.
\lstinputlisting[language=xml]{src/actspotmahasiswa/layout/fragment_progress.xml}

Selanjutnya kita buat file ProgressFragment.java di package fragment. Kemudian tambahkan script berikut di file tersebut.
\lstinputlisting[language=Java]{src/actspotmahasiswa/fragment/ProgressFragment.java}

Setelah itu,  tambahkan script berikut di file tersebut pada file AndroidManifest.xml
\lstinputlisting[language=xml]{src/actspotmahasiswa/AndroidManifest.xml}
